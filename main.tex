\documentclass[12pt]{article}
\usepackage[a4paper, margin=1in]{geometry}
\usepackage{titlesec}
\usepackage{enumitem}
\usepackage{setspace}
\usepackage{hyperref}
\usepackage{graphicx}
\usepackage{amsmath}
\bibliographystyle{IEEEtran}
\usepackage{lmodern}

\setstretch{1.3}
\titleformat{\section}{\large\bfseries}{\thesection}{1em}{}
\titleformat{\subsection}{\normalsize\bfseries}{\thesubsection}{1em}{}

\title{\textbf{Design of a Residential Floor Wiring System Using AutoCAD}}
\author{}
\date{}

\begin{document}

\maketitle

\section*{Introduction}
Electricity is an essential part of modern life. From the lights in our rooms to the devices we rely on daily, a safe and functional electrical system supports nearly everything we do. However, while using electricity is simple and routine, the systems behind it are anything but. Every switch, socket, fan, and fixture must be strategically planned, installed, and connected to ensure both safety and convenience.

Electrical layout design is one of the most crucial components of building infrastructure, yet it often goes unnoticed. This project allowed me to delve into this behind-the-scenes process by designing the electrical wiring for a residential floor using AutoCAD. It was not just a software task — it was a hands-on exploration of how homes are powered, how people interact with space, and how safety and practicality guide every decision.

The layout includes typical household spaces: a bedroom, a living room, a combined kitchen and dining area, a bathroom, and a toilet (W.C.). Each room has unique requirements in terms of lighting, switch accessibility, and ventilation, which challenged me to think critically and carefully about every element in the design. 

This task emphasized the importance of understanding user behavior, room functionality, and the implications of even minor design decisions. For instance, placing a switch too close to a water source or behind a door may seem like a small issue on paper, but in real life, it can pose safety hazards or discomfort. In this way, I learned that electrical design requires not only technical skills but also empathy and foresight.

Through this project, I discovered that good design is invisible — when done correctly, it makes daily life smoother, safer, and more efficient without drawing attention to itself.

\newpage

\section*{Project Objectives}
The main aim of this task was to design a practical and realistic wiring system for a home using AutoCAD. Key goals included:

\begin{itemize}
    \item Creating a detailed floor layout with accurate room divisions.
    \item Positioning lights and fans in appropriate, functional locations.
    \item Locating switchboards at convenient and safe points.
    \item Connecting components with a simple, logical, and secure wiring scheme.
    \item Utilizing AutoCAD features such as layers, line types, and symbols effectively.
    \item Adhering to standard electrical design practices.
    \item Documenting the entire process clearly and honestly.
\end{itemize}

\newpage

\section*{Wiring Layout Diagram}
\begin{figure}[h!]
    \centering
     \includegraphics[width=0.8\textwidth]{ad2.png}
    \caption{Residential floor wiring layout}
    \label{fig:wiring}
\end{figure}

\section*{Room-by-Room Overview}

\subsection*{Bedroom}
\begin{itemize}
    \item Lighting: Two ceiling lights near the center.
    \item Fan: One ceiling-mounted fan centrally located.
    \item Switchboard: Positioned near the entrance for convenience.
\end{itemize}

\subsection*{Living Room}
\begin{itemize}
    \item Lighting: Two lights on opposite sides of the room.
    \item Fan: A single fan at the center.
    \item Switchboard: Placed close to the main door.
\end{itemize}

\subsection*{Kitchen and Dining Area}
\begin{itemize}
    \item Lighting: Separate light points for the kitchen and dining sections.
    \item Fan: Installed above the dining space.
    \item Switchboard: Located between the kitchen and dining areas.
\end{itemize}

\subsection*{Bathroom and Toilet (W.C.)}
\begin{itemize}
    \item Lighting: One light in each area, placed centrally.
    \item Switchboard: Installed outside the wet zones to ensure safety.
\end{itemize}

\section*{Symbols and Wiring Plan}
Standard AutoCAD symbols were used to maintain clarity and consistency throughout the layout. Examples include:

\begin{itemize}
    \item \textbf{L}: Light points.
    \item \textbf{Fan Symbol}: Representing ceiling fans.
    \item \textbf{SB}: Switchboard locations.
    \item \textbf{D}: Doors for reference during switch placement.
    \item \textbf{W}: Windows to consider during wiring and ventilation planning.
\end{itemize}

\section*{Wiring Technique Applied}
\begin{itemize}
    \item A looping method was used for connecting light points to minimize cable usage.
    \item Each light and fan was assigned an individual control switch.
    \item Electrical lines were drawn with dashed lines in a separate AutoCAD layer.
    \item A concealed wiring system was implemented for safety and better aesthetics.
\end{itemize}

\section*{Safety Considerations}
\begin{itemize}
    \item Switchboards installed at a height of 4 to 4.5 feet to prevent child access.
    \item No switches were placed inside bathrooms or toilets — they are located outside.
    \item Uniform distribution of light and fans to prevent dark spots and overheating.
    \item Assumed color codes: red for live, black for neutral, green for earth.
    \item Wiring kept away from damp areas and window openings.
\end{itemize}

\section*{Key Takeaways}
\begin{itemize}
    \item Electrical layouts should be logical and user-centric.
    \item Proper planning can significantly improve usability and safety.
    \item AutoCAD is a valuable tool for visualizing and refining designs.
    \item Adherence to safety codes is more important than visual appeal.
    \item Using standardized symbols and layers helps maintain clarity in complex designs.
\end{itemize}

\section*{Suggestions for Future Improvement}
\begin{itemize}
    \item Include dedicated power outlets for major appliances.
    \item Add exhaust fans in the kitchen and bathroom.
    \item Install a distribution board with labeled circuits.
    \item Integrate inverter/backup power for essential rooms.
    \item Use LED lighting to reduce energy consumption.
\end{itemize}

\newpage

\section*{Conclusion}
This project was not just about creating a layout in AutoCAD — it served as a valuable lesson in how critical electrical design is in modern living. It opened my eyes to the amount of planning, responsibility, and attention to detail required to build a safe and efficient electrical system for a home.

Simple actions like flipping a switch involve much more behind the scenes than we typically realize. Each wire path, switch position, and socket location must be thought out carefully with both safety and usability in mind. I learned that effective wiring design is not only a technical task but also one that requires understanding how people live and interact with their environment.

AutoCAD played a central role in helping me bring my design ideas to life. While it was challenging at first, mastering the tools and features became an enjoyable process. Layer management and symbol usage significantly improved my ability to organize and present the layout clearly.

Overall, this assignment has provided me with a solid foundation in residential wiring design. I now feel better prepared to tackle more advanced projects involving load calculation, detailed circuit planning, and protective device integration. This experience has also increased my appreciation for the role electrical engineers play in ensuring our homes are functional, efficient, and most importantly — safe.

\section*{References}
\begin{itemize}
    \item IS 732:2019 — Code of Practice for Electrical Wiring Installations.
    \item AutoCAD Electrical Design Tutorials.
    \item Bansal, N.K., *Practical Wiring Techniques*.
\end{itemize}

\end{document}
