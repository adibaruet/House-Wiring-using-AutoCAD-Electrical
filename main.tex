\documentclass[12pt]{article}
\usepackage[a4paper, margin=1in]{geometry}
\usepackage{titlesec}
\usepackage{enumitem}
\usepackage{setspace}
\usepackage{hyperref}
\usepackage{graphicx}
\usepackage{amsmath}
\bibliographystyle{IEEEtran}


\usepackage{lmodern}

\setstretch{1.3}
\titleformat{\section}{\large\bfseries}{\thesection}{1em}{}
\titleformat{\subsection}{\normalsize\bfseries}{\thesubsection}{1em}{}

\title{\textbf{Design of a Residential Floor Wiring System Using AutoCAD}}
\author{}
\date{}

\begin{document}

\maketitle

\section*{Introduction}
Electricity is such an ordinary part of our daily lives that we often take it for granted. From the moment we wake up and turn on the lights, to charging our phones, using fans, or cooking dinner, we depend on a constant and safe supply of electricity. But behind every switch we flip or socket we use, there’s an entire system — planned, drawn, and carefully installed — to make it all work seamlessly.

Until this project, I never fully appreciated the thought and detail that goes into planning something as “invisible” as electrical wiring in a home. This assignment gave me the opportunity to step into the role of someone who actually designs these systems — not just thinking of how to connect wires, but also how to do it smartly and safely.

Using AutoCAD, I created a simplified residential wiring layout that includes commonly used spaces: a bedroom, living room, kitchen with a dining area, bathroom, and toilet (W.C.). Each of these rooms has different lighting and appliance needs, which meant I had to plan not just where the lights and fans would go, but also where switches should be placed so that they’re easy and safe to access.

One of the most eye-opening parts of this process was realizing how important everyday details are. For example, placing a switch on the wrong side of the door or too close to a water source can cause serious inconvenience or even danger. These small details make a big difference in real life.

Planning this layout made me see that electrical design isn’t just a technical skill — it’s a mix of logic, foresight, safety awareness, and empathy for the people who will use the space.
\newpage


\section*{Project Objectives}
The goals of this assignment were not just to draw something using software, but to understand how electrical wiring works in real life. Specific objectives include:

\begin{itemize}
    \item Draw a proper floor layout with clear room divisions.
    \item Place lights and fans in locations where they’d actually make sense.
    \item Put switchboards where they’d be easy to reach and use.
    \item Connect the components in a way that’s both simple and safe.
    \item Use AutoCAD tools like layers, symbols, and line types effectively.
    \item Follow common electrical standards wherever possible.
    \item Write a clear and honest report about what I did and learned.
\end{itemize}
\section*{Wiring Plan Diagram}

\begin{figure}[h!]
    \centering
    \includegraphics[width=0.8\textwidth]{ad.png}
    \caption{Wiring layout plan for residential floor}
    \label{fig:wiring}
\end{figure}



\section*{Room-by-Room Analysis}
\subsection*{Bedroom}
\begin{itemize}
    \item Lights: Two ceiling lights, placed near the center.
    \item Fan: One ceiling fan in the middle.
    \item Switchboard: Just inside the door for easy access.
\end{itemize}

\subsection*{Living Room}
\begin{itemize}
    \item Lights: Two lights, one on each side of the room.
    \item Fan: One ceiling fan at the center.
    \item Switchboard: Next to the main door.
\end{itemize}

\subsection*{Kitchen cum Dining}
\begin{itemize}
    \item Lights: Two lights — one for the kitchen, one for dining.
    \item Fan: Positioned over the dining area.
    \item Switchboard: Between the kitchen and dining area.
\end{itemize}

\subsection*{Bathroom and W.C.}
\begin{itemize}
    \item Lights: One each, centered.
    \item Switchboard: Placed outside for safety.
\end{itemize}

\section*{Symbols and Wiring Plan}
Standard electrical and architectural symbols were used in the AutoCAD layout. These include:

\begin{itemize}
    \item \textbf{L}: Light point (ceiling-mounted).
    \item \textbf{Fan Symbol}: Representing ceiling fans at central positions.
    \item \textbf{SB}: Switchboard locations.
    \item \textbf{D}: Doors, for reference to switch placement.
    \item \textbf{W}: Windows, to avoid wiring conflicts and plan ventilation.
\end{itemize}

These symbols ensured clarity and consistency in the plan.

\section*{Wiring Method Used}
\begin{itemize}
    \item Looping system for lights to simplify and save cable.
    \item Individual control switches for lights and fans.
    \item Dashed lines represent wiring, drawn on a separate AutoCAD layer.
    \item Concealed wiring system used for modern safety and aesthetics.
\end{itemize}

\section*{Safety Rules Followed}
\begin{itemize}
    \item Switchboards at 4–4.5 feet height, safe from children.
    \item No switches inside bathrooms or toilets — placed outside.
    \item Even distribution of lights and fans to avoid dark or hot spots.
    \item Standard color codes assumed — red (live), black (neutral), green (earth).
    \item Wiring placed away from water points and windows.
\end{itemize}

\section*{What I Learned}
\begin{itemize}
    \item Wiring must be logical and user-friendly.
    \item Planning is essential — wrong switch placement can cause discomfort.
    \item AutoCAD helps visualize and fix layout ideas.
    \item Safety and standards must come before aesthetics.
    \item Symbols and layers make complex designs manageable and readable.
\end{itemize}

\section*{Real-World Improvements I Would Add}
\begin{itemize}
    \item Power sockets for TVs, chargers, and appliances.
    \item Exhaust fans in bathrooms and kitchens.
    \item MCB box and proper circuit labels.
    \item Backup/inverter lines for essential rooms.
    \item LED lights for energy efficiency.
\end{itemize}
\newpage

\section*{Conclusion}
This project was much more than just an AutoCAD drawing task — it turned out to be a valuable and eye-opening experience about how electricity quietly powers our lives every day. When we walk into a room and flip a switch, we rarely think about the wires running behind the walls, the switches placed at convenient spots, or the way appliances are safely connected to power. Through this assignment, I learned that every little detail in an electrical layout — from the placement of a light point to the routing of a wire — is planned with intention, logic, and care.
Designing the wiring system for a residential floor made me realize how much responsibility comes with even the most basic layout. One small mistake, like placing a switch too close to water or misconnecting a wire, can cause serious safety issues. It taught me that electrical planning isn't just a technical task — it's about thinking ahead, considering how people actually move through and use a space, and always keeping safety and usability in mind.
Working with AutoCAD was also a big part of my learning. At first, I found the tools a bit overwhelming, but as I practiced more with features like blocks, polylines, and layer management, I gained confidence and started to enjoy the process. Using layers helped keep different parts of the drawing organized — like separating walls, wiring lines, and electrical symbols. This made my layout much easier to read and edit, and showed me how professionals handle complex plans in the real world.
Most importantly, I now have a strong foundation in designing residential wiring layouts — not just in theory, but in a way that’s practical and visual. In the future, I’d love to take this learning further by creating more advanced designs. For example, including electrical load calculations, adding plug point planning, or even working on three-phase wiring systems for larger buildings or commercial spaces. I also want to learn more about integrating safety devices like MCBs, RCCBs, and surge protectors into the design.
Overall, this project gave me a much deeper respect for the role of electrical engineers and designers. It taught me that wiring isn’t just about electricity — it’s about making people’s lives safer, easier, and more comfortable. And with that, I’m excited to keep learning and growing in this field.


\section*{References}
\begin{itemize}
    \item IS 732:2019 — Code of Practice for Electrical Wiring Installations.
    \item AutoCAD Electrical Design Tutorials.
    \item Practical Wiring Techniques, by N.K. Bansa.
\end{itemize}

\end{document}
